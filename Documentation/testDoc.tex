\documentclass[11pt,a4paper,oneside,notitlepage,final]{article}
\usepackage[utf8]{inputenc}

\usepackage{mathtools}
\usepackage{amsfonts,amsthm,amssymb}
\usepackage{bm}
\usepackage{enumerate}
\usepackage{hyperref}
\usepackage{color}
\usepackage[strings]{underscore}
\usepackage{booktabs}

\usepackage[
textwidth=15.2cm, 
textheight=24.3cm,
hmarginratio=1:1,
vmarginratio=1:1]{geometry}

\setlength{\parskip}{12pt}
\setlength{\parindent}{0pt}

\setlength{\mathsurround}{0.2pt}

\newcommand{\R}{\mathbb{R}}
\newcommand{\N}{\mathbb{N}}
\newcommand{\Z}{\mathbb{Z}}
\newcommand{\Q}{\mathbb{Q}}
\newcommand{\abs}[1]{ \left| #1 \right| }
%\newcommand{\abs}[1]{\lvert#1\rvert}
\newcommand{\bigabs}[1]{\bigl\lvert#1\bigr\rvert}
\newcommand{\biggabs}[1]{\biggl\lvert#1\biggr\rvert}
\newcommand{\norm}[1]{\lVert#1\rVert}
\newcommand{\ointerv}[1]{\mathopen({#1}\mathclose)}
%\newcommand{\ocinterv}[1]{\mathopen({#1}\mathclose]}
\newcommand{\ocinterv}[1]{\left( #1 \right] }
%\newcommand{\cointerv}[1]{\mathopen[{#1}\mathclose)}
\newcommand{\cointerv}[1]{\left[ #1 \right)}
\newcommand{\cinterv}[1]{ \left[ #1 \right] }
%\newcommand{\cinterv}[1]{\mathopen[{#1}\mathclose]}
\newcommand{\thalf}{\tfrac{1}{2}}
\newcommand{\half}{\frac{1}{2}}
\newcommand{\tint}{\textstyle\int}
\newcommand{\rpm}{\sbox0{$1$}\sbox2{$\scriptstyle\pm$}
	\raise\dimexpr(\ht0-\ht2)/2\relax\box2 }

\DeclareMathOperator*{\evalop}{\Big/}
\newcommand{\eval}[2]{\evalop_{\!\!\!\!\!\!#1}^{\,\,\,\,\,\,#2}}

\newcommand{\Tas}{\mathrm{Tas}}
\newcommand{\Exp}{\mathrm{Exp}}
\newcommand{\Bin}{\mathrm{Bin}}
%\newcommand{\Poi}{\mathrm{P}}
%\newcommand{\Nor}{\mathsf{N}}

% Stochastic independence symbol
\makeatletter
\def\indept{\mathchoice
	{\setbox0\hbox{$\displaystyle\m@th\perp$}\mathrel
		{\hbox to1.25\wd0{\copy0\hss\box0}}}%
	{\setbox0\hbox{$\textstyle\m@th\perp$}\mathrel
		{\hbox to1.25\wd0{\copy0\hss\box0}}}%
	{\setbox0\hbox{$\scriptstyle\m@th\perp$}\mathrel
		{\hbox to1.25\wd0{\copy0\hss\box0}}}%
	{\setbox0\hbox{$\scriptscriptstyle\m@th\perp$}\mathrel
		{\hbox to1.25\wd0{\copy0\hss\box0}}}}
\makeatother

\DeclareMathOperator{\Var}{var}
\DeclareMathOperator{\Cov}{cov}
\DeclareMathOperator{\Cor}{cor}

\newcommand{\assim}{\underset{\mathit{as}}{\sim}}

\renewcommand{\vec}[1]{{\mathbf{#1}}}

\newcounter{harjnro}

\newenvironment{harjlist}
{\begin{list}{\kor{\arabic{harjnro}.}}
		{\usecounter{harjnro}
			\setlength{\labelwidth}{0em}
			\setlength{\labelsep}{0.4em}
			\setlength{\leftmargin}{0pt}
			\setlength{\parsep}{5pt}
			\setlength{\itemindent}{0.4em}
			\setlength{\topsep}{0.5\parskip}
			\setlength{\itemsep}{0.75\parskip}
		}
	}
	{\end{list}}

\newenvironment{ratkaisu}{\kor{Ratkaisu:} \\}{}
\newenvironment{arvostelu}{\textit{\kor{Arvosteluohje: }}}{}

\newenvironment{kohdat}
{\begin{list}{\alph{enumi})}
		{\usecounter{enumi}
			\setlength{\labelwidth}{0em}
			\setlength{\labelsep}{0.5em}
			\setlength{\leftmargin}{0em}
			\setlength{\parsep}{3pt}
			\setlength{\itemindent}{0.5em}
			\setlength{\topsep}{0pt}
			\setlength{\itemsep}{0pt}
		}
	}
	{\end{list}}

\newcommand{\kor}[1]{\textbf{#1}}

\newcommand{\kaanna}{\vfill\hfill KÄÄNNÄ!\pagebreak}

\newcommand{\stcomp}[1]{{#1}^\complement}

\newcommand{\ps}{\mathcal{P}}

\pagestyle{empty}

\makeatletter
\DeclareRobustCommand{\em}{%
  \@nomath\em \if b\expandafter\@car\f@series\@nil
  \normalfont \else \bfseries \fi}
\makeatother


\begin{document}
	
	\setlength{\abovedisplayskip}{8.0pt plus3.0pt minus4.0pt}
	\setlength{\abovedisplayshortskip}{0.0pt plus3.0pt}
	\setlength{\belowdisplayskip}{8.0pt plus3.0pt minus4.0pt}
	\setlength{\belowdisplayshortskip}{7.0pt plus3.0pt minus3.0pt}
	
	\title{sievenna Testing}
	\author{Onni Aarne}
	\maketitle
	
	\section{Unit Testing}
	A link to an up-to-date report can be found in README.md.
	
	\section{Performance Testing}
	sievenna's Huffman coding has been tested on the Large Text Compression Benchmark \cite{mahoney2011large} as well as an uncompressed tarball of the standard Calgary corpus \cite{bell1989modeling}.
	
	Image compression benchmark used was an 8-bit RGB photograph called nightshot\_iso\_100 from \href{http://imagecompression.info/test_images/}{imagecompression.info} \cite{imagecompressionbenchmark}.
	
	Results achieved can be seen in table \ref{results}. The software fares best with text, achieving a compression ratio of about 1.5. The different sized variants of the Large Text Compression benchmark combined with the relatively small Calgary corpus show that the ratios achieved are not heavily dependent on input size.
	
	If the reader wishes to replicate these results, among the project's test HuffmanCoderTest and CalgaryTest test the performace of the program on the nightshot and the Calgary corpus respectively. The Large Text Compression benchmark is too large to include in the repository, but can be found \href{http://mattmahoney.net/dc/text.html}{here}.
	
	\begin{table}
		\begin{tabular}{llllll}
			\toprule
			File & Size & Compressed & Ratio & Comp. Time & Decomp. Time \\
			\midrule
			nightshot\_iso\_100.ppm & 22.128 MB & 17.843427 MB & 1.240 & 2.654 s & 1.989 s \\
			enwik8 & 100 MB & 63.862 & 1.566 & 9.100 s & 6.900 s \\
			enwik9 & 1000 MB & 648.370 MB & 1.542 & 68 s & 59 s \\
			Calgary Corpus & 3.154 MB & 2.125 MB & 1.484 & 0.385 s & 0.321 s \\
			\bottomrule
		\end{tabular}
		\caption{Performance statistics for Huffman coding. Note: enwik8 and enwik9 are different sizes of the Large Text Compression Benchmark.}
		\label{results}
	\end{table}
	
	\bibliographystyle{plain}
	\bibliography{bench.bib}
\end{document}
