\documentclass[11pt,a4paper,oneside,notitlepage,final]{article}
\usepackage[utf8]{inputenc}

\usepackage{mathtools}
\usepackage{amsfonts,amsthm,amssymb}
\usepackage{bm}
\usepackage{enumerate}
\usepackage{hyperref}
\usepackage{color}

\usepackage[
textwidth=15.2cm, 
textheight=24.3cm,
hmarginratio=1:1,
vmarginratio=1:1]{geometry}

\setlength{\parskip}{12pt}
\setlength{\parindent}{0pt}

\setlength{\mathsurround}{0.2pt}

\newcommand{\R}{\mathbb{R}}
\newcommand{\N}{\mathbb{N}}
\newcommand{\Z}{\mathbb{Z}}
\newcommand{\Q}{\mathbb{Q}}
\newcommand{\abs}[1]{ \left| #1 \right| }
%\newcommand{\abs}[1]{\lvert#1\rvert}
\newcommand{\bigabs}[1]{\bigl\lvert#1\bigr\rvert}
\newcommand{\biggabs}[1]{\biggl\lvert#1\biggr\rvert}
\newcommand{\norm}[1]{\lVert#1\rVert}
\newcommand{\ointerv}[1]{\mathopen({#1}\mathclose)}
%\newcommand{\ocinterv}[1]{\mathopen({#1}\mathclose]}
\newcommand{\ocinterv}[1]{\left( #1 \right] }
%\newcommand{\cointerv}[1]{\mathopen[{#1}\mathclose)}
\newcommand{\cointerv}[1]{\left[ #1 \right)}
\newcommand{\cinterv}[1]{ \left[ #1 \right] }
%\newcommand{\cinterv}[1]{\mathopen[{#1}\mathclose]}
\newcommand{\thalf}{\tfrac{1}{2}}
\newcommand{\half}{\frac{1}{2}}
\newcommand{\tint}{\textstyle\int}
\newcommand{\rpm}{\sbox0{$1$}\sbox2{$\scriptstyle\pm$}
	\raise\dimexpr(\ht0-\ht2)/2\relax\box2 }

\DeclareMathOperator*{\evalop}{\Big/}
\newcommand{\eval}[2]{\evalop_{\!\!\!\!\!\!#1}^{\,\,\,\,\,\,#2}}

\newcommand{\Tas}{\mathrm{Tas}}
\newcommand{\Exp}{\mathrm{Exp}}
\newcommand{\Bin}{\mathrm{Bin}}
%\newcommand{\Poi}{\mathrm{P}}
%\newcommand{\Nor}{\mathsf{N}}

% Stochastic independence symbol
\makeatletter
\def\indept{\mathchoice
	{\setbox0\hbox{$\displaystyle\m@th\perp$}\mathrel
		{\hbox to1.25\wd0{\copy0\hss\box0}}}%
	{\setbox0\hbox{$\textstyle\m@th\perp$}\mathrel
		{\hbox to1.25\wd0{\copy0\hss\box0}}}%
	{\setbox0\hbox{$\scriptstyle\m@th\perp$}\mathrel
		{\hbox to1.25\wd0{\copy0\hss\box0}}}%
	{\setbox0\hbox{$\scriptscriptstyle\m@th\perp$}\mathrel
		{\hbox to1.25\wd0{\copy0\hss\box0}}}}
\makeatother

\DeclareMathOperator{\Var}{var}
\DeclareMathOperator{\Cov}{cov}
\DeclareMathOperator{\Cor}{cor}

\newcommand{\assim}{\underset{\mathit{as}}{\sim}}

\renewcommand{\vec}[1]{{\mathbf{#1}}}

\newcounter{harjnro}

\newenvironment{harjlist}
{\begin{list}{\kor{\arabic{harjnro}.}}
		{\usecounter{harjnro}
			\setlength{\labelwidth}{0em}
			\setlength{\labelsep}{0.4em}
			\setlength{\leftmargin}{0pt}
			\setlength{\parsep}{5pt}
			\setlength{\itemindent}{0.4em}
			\setlength{\topsep}{0.5\parskip}
			\setlength{\itemsep}{0.75\parskip}
		}
	}
	{\end{list}}

\newenvironment{ratkaisu}{\kor{Ratkaisu:} \\}{}
\newenvironment{arvostelu}{\textit{\kor{Arvosteluohje: }}}{}

\newenvironment{kohdat}
{\begin{list}{\alph{enumi})}
		{\usecounter{enumi}
			\setlength{\labelwidth}{0em}
			\setlength{\labelsep}{0.5em}
			\setlength{\leftmargin}{0em}
			\setlength{\parsep}{3pt}
			\setlength{\itemindent}{0.5em}
			\setlength{\topsep}{0pt}
			\setlength{\itemsep}{0pt}
		}
	}
	{\end{list}}

\newcommand{\kor}[1]{\textbf{#1}}

\newcommand{\kaanna}{\vfill\hfill KÄÄNNÄ!\pagebreak}

\newcommand{\stcomp}[1]{{#1}^\complement}

\newcommand{\ps}{\mathcal{P}}

\pagestyle{empty}

\makeatletter
\DeclareRobustCommand{\em}{%
  \@nomath\em \if b\expandafter\@car\f@series\@nil
  \normalfont \else \bfseries \fi}
\makeatother


\begin{document}
	
	\setlength{\abovedisplayskip}{8.0pt plus3.0pt minus4.0pt}
	\setlength{\abovedisplayshortskip}{0.0pt plus3.0pt}
	\setlength{\belowdisplayskip}{8.0pt plus3.0pt minus4.0pt}
	\setlength{\belowdisplayshortskip}{7.0pt plus3.0pt minus3.0pt}
	
	\title{sievenna Development Log}
	\author{Onni Aarne}
	\maketitle
	
	\section{First Week}
	\emph{Thursday 21.12} \\
	Locked down the topic and named the project.\\
	Created this document and the GitHub repository.\\
	Wrote a basic draft of the specification document.\\
	Wrote this log.\\
	Registered on labtool.
	
	Next week I'll start to actually write the program.
	
	This week I spent maybe 5h on the project.
	
	\section{Second Week}
	I did project setup sporadically throughout the Holidays.
	
	As I traveled on 27.12 I started researching the actual algorithms.\\
	Learned some stuff about information theory and the theory of compression.\\
	Kinda learned how Huffman coding works.\\
	Need to look deeper into all of that stuff.\\
	Decided to implement a simple byte-level Huffman coding algorithm as a first step.
	
	Over the next two days I worked on implementing said functionality.\\
	Ran into difficulties with bit-level I/O.

	I've been terrible about organizing my time so it's a bit tough to say how many hours I've spent on this. Traveling and the Holidays certainly haven't helped with that.\\
	I've spent at least 10 hours on this project this week.
	
	With a few hours before the deadline I'll try to learn how to unit test and implement some tests for incomplete functionality. Yay.\\
	Wrote tests for my HuffNode class. Next I need to figure out how to test my I/O class.\\
	Next week I'll implement decoding the Huffman-coded files.
\end{document}
